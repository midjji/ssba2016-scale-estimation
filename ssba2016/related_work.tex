\section{Related Work}
In the pinhole camera model, all light rays pass through the image plane and converge on the camera center. If both the image plane and the constraint that all rays pass through a common point are removed, the new camera model is referred to as \textit{generalized}. For some generalized camera configurations it is possible to recover the scale and thus translation distance from six points observed at two instances in time, and several methods have been developed. The general minimal case solver by Stewenius et al uses Gr\"obner basis and results in up to 64 solutions~\cite{stewenius2005solutions}. This theoretically appealing solver comes at the cost of computational efficiency and requires high-precision floating point types. 

Li et al argue that small rotations are likely in many applications and present a method which relies on linearizing the rotation~\cite{li2008linear}. However, the method requires 17 correspondences, which is infeasible in a RANSAC framework. 

Kneip and Li~\cite{kneip2014efficient} developed a 7-point eigenvalue minimization method. Furthermore, Ventura et al, developed an efficient and robust solver for a first-order approximation of the relative pose~\cite{ventura2015efficient}. Despite an impressive improvement in computational performance this method is still slow relative to essential matrix solvers. 

The methods above are all or nothing affairs, returning either a full relative pose or no information at all.

Finally there is the 5+1 method of Clipp~\cite{clipp2008robust}. This two-step approach first estimates the essential matrix with a five-point minimal solver wrapped in a RANSAC loop, with point correspondences from a single camera. Given the rotation and translation as given by the essential matrix decomposition the scale is computed from single correspondence in the second camera. From choosing this approach it follows that the possible degeneracies of both steps are kept separate. If the first step succeeds but the second fails, at least the camera rotation and translation direction will be available.  We accidentally rediscovered this method before doing a proper literature study and felt it was a good idea. A different but equivalent theoretical derivation follows in the theory section while implementation details and considerations are described in the method section. 

